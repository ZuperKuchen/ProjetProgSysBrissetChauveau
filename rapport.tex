\documentclass[10pt,a4paper]{article} % KOMA-Script article scrartcl
\usepackage{lipsum}
\usepackage[frenchb]{babel}
\usepackage[utf8x]{inputenc}
\usepackage[T1]{fontenc}
\usepackage{url}
\usepackage{minted}
\usepackage[nochapters]{classicthesis} 

\begin{document}
    \title{\rmfamily\normalfont\spacedallcaps{Projet Programmation Systeme}}
    \author{\spacedlowsmallcaps{Pierre Chauveau, Remi Brisset, Francois Audoy}}
    \date{\today} % no date
    
    \maketitle

    \begin{abstract}
      ici on fait l'introduction / résumé du projet
    \end {abstract}
    
    \tableofcontents
    
    \section{Sauvegarde et chargemement des cartes}

    test des accents déjà où ça
    Ecrire ici le contenue de la première partie
    \subsection{Sauvegarde}
    Idem
    
    \subsection{Chargement}
    Idem
    
    \subsection{Utilitaire de manipulation de carte}
    Pour nous assister dans nos tests, nous avons implémenté un executable readMap affichant sur la sortie standard le contenu d'un \emph{file} issu d'une map sauvegardée.
    Exemple d'utilisation:
    
    \subsubsection{Une erreur surprenante}
    idem
    
    \section{Gestion des temporisateurs}
    Idem
    
    \subsection{Choix d'implémentation}
    idem

    \subsection{autre sous section}
    ide
    
\end{document}
